\documentclass[a4paper, 10pt]{article}

\usepackage[utf8]{inputenc}
\usepackage{fancyhdr}
\pagestyle{fancy}
\setlength{\headheight}{24.0pt}

\rhead{
	\begin{tabular}{r}
		Design and Graphics in Game Development\\
	\end{tabular}
}

\lhead{
	\begin{tabular}{l}
		SS 15\\
	\end{tabular}
}

\begin{document}

	\section{Proposal Mini-Game}
	
	\subsection{Group name and members}	
	
	Group name: \textbf{Flynncade}	
	\\
	\\
	\begin{tabular}{rl}
		Christoph Reinhart  & christoph.reinhart@students.unibe.ch\\
		Raphaël Tuor  &\\
		Nicolas Spycher & nicolas.spycher@students.unibe.ch
	\end{tabular}
	
	\subsection{Theme}
	\par{We are thinking about making a Tron-like game, because the graphic workload would be reduced without compromising a good look and feel of the game.}
	
	\subsection{The idea}
	\par{The main idea behind our mini-game is that the player can change dimensions. There would be puzzle which he has to solve either in a 2D side view or in a 2.5D overview. Some obstacles would only be passable in a certain view.}
	
	\subsection{Tetrad}
	\vspace{10pt}
	\begin{tabular}{ l | l | p{5cm} | p{2cm}}
		\textbf{Technology} & \textbf{Aestetics} & \textbf{Mechanics} & \textbf{Story} \\ \hline
		PC & Tron-like & Jump and Run (switch dimensions), solve puzzles (for example command promt), get from A to B, kill enemies, switch between two players, some puzzles have to be solved together & Tron-like \\
	\end{tabular}
	
	\subsection{Planning}
	\vspace{10pt}
	\begin{tabular}{rl}
		End of march & first prototype with first level, only some mechanics and physics \\
		End of april & finished in terms of mechanics. Add textures and more levels\\
		May & Finishing touch, balancing, gameplay, maybe more levels\\
	\end{tabular}

\end{document}